\begin{document}
For the generation of this plot we calculated and evaluated 50 background potentials from identicals sets of $N_{scatt} =4000$, $V_{max} = 0.1$ and the respective $\sigma$. For the potentials with the largest and smallest correlation length, we show the correlation lengths as calculated from a 1D fit of the ACF, and the two width $\sigma_{x,y}$ as calculated from a 2D gaussian fit to the ACF in the left pane. The right pane shows $\frac{\sigma_{x}{\sigma_{y}}$ for the largest and smallest value of $l_{corr}$. Note that a value of 1 corresponds to perfectly symmetric gaussian fit to the 2D ACF.
  Since the autocorrelation function of an isolated gaussian is a gaussian itself, we may attribute the observed deviation from this perfect shape for impurities wider than 0.4 $l_{0}$ (in this case) to the overlapp of the individual scatterer, corresponding to the increasing spread of the $L_{corr}(\sigma)$ as shown in Figs.\ref{}   \footnote{Comparing this value with the geometrical interpreation given above, we may conclude that a significant overlapp starts at five times larger values for $N_{scatt}$ }. 
\end{document}
